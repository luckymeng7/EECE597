
\noindent The following individuals certify that they have read,
and recommend to the Faculty of Graduate and Postdoctoral Studies
for acceptance, the thesis entitled:

\begin{center}
{\large \textbf{Trajectory planning with real time vision-based obstacle detection}}
\end{center}

submitted by \textbf{Xuemeng Li} in partial fulfillment of the requirements for the degree of \textbf{Master of Engineering} in \textbf{Electrical and Computer Engineering}.%
\par\bigskip%

\noindent\textbf{Examining Committee:}%
\par\medskip\noindent{Maryam Kamgarpour, Electrical and Computer Engineering, UBC}\\\emph{Supervisor}
\par\medskip\noindent{Mahdi Yousefi, Avestec Technologies Inc.}\\\emph{Supervisor}
\cleardoublepage

\chapter{Abstract}

Autonomous aerial vehicles are broadly used to assist human in dangerous or complex monitoring tasks, such as inspection of hard to reach high voltage power lines and oil pipes, monitoring and distinguishing wildfire, and improving precision farming. An autonomous navigation system can support drones or robots to move towards targeted positions without any external control. A fully functional autonomous system requires an integration on a wide range of algorithms, including trajectory planning algorithms, object detection with image processing and robust control algorithms for path following. 

In this report, an implementation of trajectory planning with object detected from the video with depth information has been described. The locations and sizes of the obstacles are determined through image processing and convex optimization with the depth information collected from Intel RealSense Camera. Then the trajectory planning function based on Rapid-exploring random tree (RRT) algorithm will plan the path on the map which combined the detected obstacles. This process would ensure drones or robots reaching the destination domain safely.



