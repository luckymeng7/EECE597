% Options for packages loaded elsewhere
\PassOptionsToPackage{unicode}{hyperref}
\PassOptionsToPackage{hyphens}{url}
%
\documentclass[
  oneside]{ubcthesis}
\usepackage{lmodern}
\usepackage{amssymb,amsmath}
\usepackage{ifxetex,ifluatex}
\ifnum 0\ifxetex 1\fi\ifluatex 1\fi=0 % if pdftex
  \usepackage[T1]{fontenc}
  \usepackage[utf8]{inputenc}
  \usepackage{textcomp} % provide euro and other symbols
\else % if luatex or xetex
  \usepackage{unicode-math}
  \defaultfontfeatures{Scale=MatchLowercase}
  \defaultfontfeatures[\rmfamily]{Ligatures=TeX,Scale=1}
\fi
% Use upquote if available, for straight quotes in verbatim environments
\IfFileExists{upquote.sty}{\usepackage{upquote}}{}
\IfFileExists{microtype.sty}{% use microtype if available
  \usepackage[]{microtype}
  \UseMicrotypeSet[protrusion]{basicmath} % disable protrusion for tt fonts
}{}
\makeatletter
\@ifundefined{KOMAClassName}{% if non-KOMA class
  \IfFileExists{parskip.sty}{%
    \usepackage{parskip}
  }{% else
    \setlength{\parindent}{0pt}
    \setlength{\parskip}{6pt plus 2pt minus 1pt}}
}{% if KOMA class
  \KOMAoptions{parskip=half}}
\makeatother
\usepackage{xcolor}
\IfFileExists{xurl.sty}{\usepackage{xurl}}{} % add URL line breaks if available
\IfFileExists{bookmark.sty}{\usepackage{bookmark}}{\usepackage{hyperref}}
\hypersetup{
  pdftitle={Trajectory planning with real time vision-based obstacle detection},
  pdfauthor={Xuemeng Li},
  hidelinks,
  pdfcreator={LaTeX via pandoc}}
\urlstyle{same} % disable monospaced font for URLs
\usepackage{longtable,booktabs}
% Correct order of tables after \paragraph or \subparagraph
\usepackage{etoolbox}
\makeatletter
\patchcmd\longtable{\par}{\if@noskipsec\mbox{}\fi\par}{}{}
\makeatother
% Allow footnotes in longtable head/foot
\IfFileExists{footnotehyper.sty}{\usepackage{footnotehyper}}{\usepackage{footnote}}
\makesavenoteenv{longtable}
\setlength{\emergencystretch}{3em} % prevent overfull lines
\providecommand{\tightlist}{%
  \setlength{\itemsep}{0pt}\setlength{\parskip}{0pt}}
\setcounter{secnumdepth}{5}
\previousdegree{B.A.Sc., Simon Fraser University, 2017}

\usepackage{booktabs}
\usepackage{makeidx}
\usepackage{graphicx}
\usepackage{makecell}



\makeindex
\frontmatter
\usepackage[]{natbib}
\bibliographystyle{apalike}

\title{Trajectory planning with real time vision-based obstacle detection}
\author{Xuemeng Li}
\date{April 2020}

\begin{document}
\maketitle


\noindent The following individuals certify that they have read,
and recommend to the Faculty of Graduate and Postdoctoral Studies
for acceptance, the thesis entitled:

\begin{center}
{\large \textbf{Trajectory planning with real time vision-based obstacle detection}}
\end{center}

submitted by \textbf{Xuemeng Li} in partial fulfillment of the requirements for the degree of \textbf{Master of Engineering} in \textbf{Electrical and Computer Engineering}.%
\par\bigskip%

\noindent\textbf{Examining Committee:}%
\par\medskip\noindent{Maryam Kamgarpour, Electrical and Computer Engineering, UBC}\\\emph{Supervisor}
\par\medskip\noindent{Mahdi Yousefi, Avestec Technologies Inc.}\\\emph{Supervisory Committee Member}
\cleardoublepage

\chapter{Abstract}

Autonomous aerial vehicles are broadly used to assist human in dangerous or complex monitoring tasks, such as inspection of hard to reach high voltage power lines and oil pipes, monitoring and distinguishing wildfire, or improving precision farming. An autonomous navigation system can support the drones or robots to move towards the goal without any external control. A fully functional autonomous system requires an integration on a wide range of algorithms, including trajectory planning algorithms, object detection with image processing and robust control algorithms for path following. 

In this project, implementation and evaluation of the effectiveness of trajectory planning Rapid-exploring random tree (RRT) algorithm with object detected from the depth information collected from Intel Real Sense camera has been described. The location and size of the targets would be determined with image processing. And the trajectory planning algorithm RRT will plan the path in real time that ensure the safety of reaching the desired target domain.

{
\setcounter{tocdepth}{2}
\tableofcontents
}
\hypertarget{list-of-abbreviations}{%
\chapter{List of Abbreviations}\label{list-of-abbreviations}}

\begin{longtable}{ll}
\toprule
Short & Long\\
\midrule
CIHR & Canadian Institutes of Health Research\\
MSFHR & Michael Smith Foundation for Health Research\\
NCI & National Cancer Institute\\
PMCOA & PubMed Central Open Access\\
WGS & whole genome sequencing\\
\bottomrule
\end{longtable}

\hypertarget{acknowledgements}{%
\chapter{Acknowledgements}\label{acknowledgements}}

I would like to thank the supervision of Dr.~Maryam Kamgarpour and Dr.~Mahdi Yousefi.

\mainmatter

\hypertarget{introduction}{%
\chapter{Introduction}\label{introduction}}

Here is the introduction to this thesis. I will do several things. It is heavily based on previous seminal work on penguins \citep{meyer2003pressures}.

\hypertarget{first-chapter-of-research}{%
\chapter{First Chapter of Research}\label{first-chapter-of-research}}

Here is a research chapter. Look a graph in Figure \ref{fig:myfigure}.



Also check out the associated data in Table \ref{tab:mytable}.

\hypertarget{conclusions}{%
\chapter{Conclusions}\label{conclusions}}

I did several things and will now discuss why they are good.

  \bibliography{thesis.bib}

\end{document}
